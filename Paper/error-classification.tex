There exists a significant body of work in the area of text error classification in MT, and proposed methods are primarily divided into manual or automatic. In the former category, approaches describe detailed error typologies to be used in conducting a systematic error annotation \cite{fishel2011automatic,vilar2006error, lommel2014multidimensional, farrus2010linguistic, costa2015linguistically}. On the other hand,  automatic error classification tools \cite{zeman2011addicter, popovic2011hjerson,popovic2015poor} detect predefined error types, by measuring the alignment of the output sentence of a MT system and its reference translation. Further work in this direction incorporates the idea that a certain sentence error can fit in more than one categories \cite{klubivcka2018quantitative, lommel2014assessing} and extends previous works to multi-label and multi-error automatic classification \citep{popovic2019automatic}. 

Most recently, error detection and classification are examined as applications of neural sequence labeling methods, which aim to detect errors and predict their surrounding context \cite{rei2017semi}. The latest methods utilize powerful contextual word embeddings in an unsupervised setting \cite{bell2019context}, and overcome the disadvantages of older ones requiring large amounts of labeled data to perform efficiently.